\documentclass[diss-proposta,nocipinfo]{texufpel}

\usepackage[utf8]{inputenc} % acentuacao
\usepackage{graphicx} % para inserir figuras
\usepackage[T1]{fontenc}
\usepackage{todo}
\usepackage{xcolor, colortbl, color}

\hypersetup{
    hidelinks, % Remove coloração e caixas
    unicode=true,   %Permite acentuação no bookmark
    linktoc=all %Habilita link no nome e página do sumário
}

\unidade{Centro de Desenvolvimento Tecnológico}
\programa{Programa de Pós-Graduação em Computação}
\curso{Ciência da Computação}

\title{Escalonador de Memórias Transacionais para arquiteturas NUMA}

\author{Costa}{Michael Alexandre}
\advisor[Prof.~Dr.]{Du Bois}{André Rauber}
\coadvisor[Prof.~Dr.]{Pilla}{Mauricio Lima}

%Palavras-chave em PT_BR
\keyword{Escalonador}
\keyword{Memórias Transacionais}
\keyword{Arquiteturas NUMA}

%Palavras-chave em EN_US
\keywordeng{Scheduler}
\keywordeng{Transactional Memory}
\keywordeng{NUMA Architecture}

\begin{document}

\maketitle
\sloppy

%Resumo em Portugues (no maximo 1 pagina)
\begin{abstract}
  Apresentar aqui uma breve Introdução ao Problema que está se
  pretendendo resolver ou abordar. Além disso, nesta seção,
  apresenta(m)-se o(s) principal(is) objetivo(s) do projeto e,
  portanto, a(s) principal(is) contribuição(ções).
\end{abstract}

\chapter{Motivação}
% (ENTRE 1 e 2 PÁGINAS)

Nesta seção, apresenta-se um breve histórico da área de concentração
da Dissertação, partindo do tema mais abrangente até chegar
especificamente no assunto do Projeto. Além disso, apresenta-se a
justificativa para a realização do trabalho, sua importância acadêmica
ou para comunidade e grau de inovação. Poderá também apresentar as
distinções entre o trabalho atual e outros trabalhos já realizados.

\chapter{Objetivos e Resultados}
% (ENTRE 1 e 3 PÁGINAS)

Nesta seção, apresentam-se o objetivo Geral e os objetivos Específicos
da dissertação. Os objetivos não devem ser confundidos com as
atividades. Para a definição das atividades, deve-se partir dos
objetivos determinados nesta seção. O objetivo Geral do Projeto
necessariamente deve ser algum resultado prático (implementação) ou
teórico (modelos formais ou especificações ou validações) produto da
pesquisa realizada no período do Projeto. Assim como os objetivos
específicos, que são considerados como sub-produtos do Objetivo
Geral. Além disso, deve-se apresentar os principais resultados
esperados do desenvolvimento desta dissertação.

\chapter{Metodologia}
% (ENTRE 1 e 3 PÁGINAS)

Nesta seção, apresenta-se a metodologia proposta para o
desenvolvimento da Dissertação. O proponente deve descrever as
atividades necessárias para a conclusão dos objetivos propostos.

\chapter{Cronograma}

\begin{enumerate}
  \item Revisar a bibliografia relacionada, revisando os principais escalonadores implementados e suas característica;
  \item Estudar o código fonte do escalonador base para implementação da heurística Numa-Aware;
  \item Realizar as primeira modificações na implementação do escalonador base;
  \item Escrever a fundamentação teórica da dissertação;
  \item Seminário de andamento;
  \item Implementação dos testes de validação do escalonador;
  \item Implementar as modificações no escalonador com validações utilizando o conceito Numa-Aware;
  \item Realizar e verificar os testes finais para avaliação dos dados coletados;
  \item Escrever a dissertação final;
  \item Entrega da dissertação final.
\end{enumerate}


\definecolor{intnull}{RGB}{0,0,0}
\begin{table}[h]
\centering
\caption{Cronograma de atividades do mestrado}
% \label{ }
\begin{tabular}{c|c|c|c|c|c|c|c|c|c|c|c|c}
 \hline
  & Jan & Fev & Mar & Abr & Mai & Jun & Jul & Ago & Set & Out & Nov & Dez \\
 \hline
 1 & \cellcolor{gray} & \cellcolor{gray} & & & & & & & & & & \\
 \hline
 2 & & \cellcolor{gray} & \cellcolor{gray} & & & & & & & & &  \\
 \hline
 3 & & \cellcolor{gray} & \cellcolor{gray} & \cellcolor{gray} & & & & & & & & \\
 \hline
 4 & & & & \cellcolor{gray} & \cellcolor{gray} & & & & & & & \\
 \hline
 5 & & & & & & \cellcolor{gray} & & & & & & \\
 \hline
 6 & & & & & & & \cellcolor{gray} & & & & \\
 \hline
 7 & & & & & & & \cellcolor{gray} & \cellcolor{gray} & \cellcolor{gray} & & & \\
 \hline
 8 & & & & & & & & \cellcolor{gray} & \cellcolor{gray} & \cellcolor{gray} & & \\
 \hline
 9 & & & & & & & & & & \cellcolor{gray} & \cellcolor{gray} & \cellcolor{gray} \\
 \hline
 10 & & & & & & & & & & & & \cellcolor{gray} \\
 \hline
\end{tabular}
\end{table}


\bibliography{bibliografia}
\bibliographystyle{abnt}

\chapter{Assinaturas}
\vspace{2cm}

\begin{center}
\rule{8cm}{.3mm}
\medskip

	Michael Alexandre Costa\\
	Proponente

\end{center}

\vspace{4cm}

\begin{center}
\rule{8cm}{.3mm}
\medskip

	André R. Du Bois\\
	Prof. Orientador

\end{center}
\end{document}
