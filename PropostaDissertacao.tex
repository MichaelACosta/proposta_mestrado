\documentclass[diss-proposta,nocipinfo]{texufpel}

\usepackage[utf8]{inputenc} % acentuacao
\usepackage{graphicx} % para inserir figuras
\usepackage[T1]{fontenc}
\usepackage{todo}
\usepackage{xcolor, colortbl, color}

\hypersetup{
    hidelinks, % Remove coloração e caixas
    unicode=true,   %Permite acentuação no bookmark
    linktoc=all %Habilita link no nome e página do sumário
}

\unidade{Centro de Desenvolvimento Tecnológico}
\programa{Programa de Pós-Graduação em Computação}
\curso{Ciência da Computação}

\title{Escalonador de Memórias Transacionais para arquiteturas NUMA}

\author{Costa}{Michael Alexandre}
\advisor[Prof.~Dr.]{Du Bois}{André Rauber}
\coadvisor[Prof.~Dr.]{Pilla}{Mauricio Lima}

%Palavras-chave em PT_BR
\keyword{Escalonador}
\keyword{Memórias Transacionais}
\keyword{Arquiteturas NUMA}

%Palavras-chave em EN_US
\keywordeng{Scheduler}
\keywordeng{Transactional Memory}
\keywordeng{NUMA Architecture}

\begin{document}

\maketitle
\sloppy

%Resumo em Portugues (no maximo 1 pagina)
\begin{abstract}
  Apresentar aqui uma breve Introdução ao Problema que está se
  pretendendo resolver ou abordar. Além disso, nesta seção,
  apresenta(m)-se o(s) principal(is) objetivo(s) do projeto e,
  portanto, a(s) principal(is) contribuição(ções).
\end{abstract}

\chapter{Motivação}
% (ENTRE 1 e 2 PÁGINAS)

Memórias Transacionais~(MT) são mecanismos de sincronização que realizam execuções atômicas e isoladas de partes compartilhadas do código. Na programação utilizando \emph{Software Transactional Memory}~(STM), o acesso à memória compartilhada é realizado dentro de uma transação executada atomicamente~\cite{teixeira15}. As Memórias transacionais propiciam aos programadores maior facilidade de desenvolver programas paralelos, sem ter de se preocupar com aquisições e liberações de \emph{locks}, assim, evitando problemas como o \emph{deadlock}.

Para garantir a sincronização das transações, STMs utilizam um sistema com detecção de conflitos e gerenciadores de contenção. Estes auxiliam na sincronização e consistência do sistema, ou seja, se duas transações manipulam o mesmo dado, a STM detecta um conflito com base em seu sistema de detecção de conflito que toma decisão de cancelar uma das transações, e reinicia a execução desta de acordo com seu gerenciador de contenção.

Os gerenciadores de contenção garantem a reexecução da transação, mas atuam após a ocorrência de um conflito, assim, não evitando que os conflitos ocorram. Outro fator desfavoral ao sistema atual de gerenciadores de conflitos é que suas regras em geral, não evitam que novos conflitos ocorram em transações já conflitadas.

Buscando reduzir o \emph{overhead} gerado por conflitos, trabalhos recentes de STM utilizam escalonadores com caracteristicas especificas para reduzir o número de conflitos. Trabalhos recentes utilizam heuristicas distintas para auxiliar na tomada de decisão do escalonador, onde, este pode decidir por inciar ou não uma \emph{thread} para evitar que conflitos ocarram.

A utilização de escalonadores de STM permitem maior controle sobre as transações em execução, possibilitando a serialização de parte do código para evitar conflitos, ou até mesmo controlar o fluxo de transações em relação ao número de \emph{cores} de uma máquina. Porem, o escalonador funciona com pasa em dados coletados ou previamente passados para ele, isto ocorre de acordo com a heuristica utilizada junto ao escalonador.

Existem heuristicas de escalonadores distintas, que influenciam diretamente na proposta de escalonamento utilizada. O trabalho~\cite{sanzo17}, apresenta uma categorização de escalonadores, com base em escalonadores existentes na bibliografia, analizando as principais heurísticas utilizadas e suas configurações.

% explicar heuristicas

O escalonador Shrink, apresentado em ~\cite{dragojevic09} e abordado no trabalho de~\cite{sanzo17}, possui heuristica baseada em previsão com base na heuristica baseada em \emph{feedback} do escalonador ATS~\cite{yoo08}. Shrink utiliza de um contador de intencidade IC, no qual com base nas ocorrencias de conflitos toma a decisão de iniciar uma transação ou aguardar outra ser executada...

O escalonador Shrink é desenvolvido junto com a biblioteca de STM \emph{tinySTM}, e assim como demais escalonares vistos na bibliografia são desenhados para utilizar arquiteturas UMA~\emph{(Uniform Memory Acess)}, assim sendo, consideram informações da execução do algoritmo e abstraem as caracteristicas da arquitetura utilizada, assumindo o uso de arquiteturas UMA.

A utilização de arquteturas NUMA~\emph{(Nom-Uniform Memory Acess)}, tem importante contribuição para computação paralela e mostranse promissoras para ampliar a utilização de \emph{cores} por máquina. Isto se da pela sua principal diferença entre as arquiteturas UMA, esta sendo o uso de barramento distintos entre o acesso à memória.

% explicar NUMA

% proposta

\chapter{Objetivos e Resultados}
% (ENTRE 1 e 3 PÁGINAS)

Nesta seção, apresentam-se o objetivo Geral e os objetivos Específicos
da dissertação. Os objetivos não devem ser confundidos com as
atividades. Para a definição das atividades, deve-se partir dos
objetivos determinados nesta seção. O objetivo Geral do Projeto
necessariamente deve ser algum resultado prático (implementação) ou
teórico (modelos formais ou especificações ou validações) produto da
pesquisa realizada no período do Projeto. Assim como os objetivos
específicos, que são considerados como sub-produtos do Objetivo
Geral. Além disso, deve-se apresentar os principais resultados
esperados do desenvolvimento desta dissertação.

\chapter{Metodologia}
% (ENTRE 1 e 3 PÁGINAS)

Nesta seção, apresenta-se a metodologia proposta para o
desenvolvimento da Dissertação. O proponente deve descrever as
atividades necessárias para a conclusão dos objetivos propostos.

\chapter{Cronograma}

\begin{enumerate}
  \item Revisar a bibliografia relacionada, revisando os principais escalonadores implementados e suas característica;
  \item Estudar o código fonte do escalonador base para implementação da heurística Numa-Aware;
  \item Realizar as primeira modificações na implementação do escalonador base;
  \item Escrever a fundamentação teórica da dissertação;
  \item Seminário de andamento;
  \item Implementação dos testes de validação do escalonador;
  \item Implementar as modificações no escalonador com validações utilizando o conceito Numa-Aware;
  \item Realizar e verificar os testes finais para avaliação dos dados coletados;
  \item Escrever a dissertação final;
  \item Entrega da dissertação final.
\end{enumerate}


\definecolor{intnull}{RGB}{0,0,0}
\begin{table}[h]
\centering
\caption{Cronograma de atividades do mestrado}
% \label{ }
\begin{tabular}{c|c|c|c|c|c|c|c|c|c|c|c|c}
 \hline
  & Jan & Fev & Mar & Abr & Mai & Jun & Jul & Ago & Set & Out & Nov & Dez \\
 \hline
 1 & \cellcolor{gray} & \cellcolor{gray} & & & & & & & & & & \\
 \hline
 2 & & \cellcolor{gray} & \cellcolor{gray} & & & & & & & & &  \\
 \hline
 3 & & \cellcolor{gray} & \cellcolor{gray} & \cellcolor{gray} & & & & & & & & \\
 \hline
 4 & & & & \cellcolor{gray} & \cellcolor{gray} & & & & & & & \\
 \hline
 5 & & & & & & \cellcolor{gray} & & & & & & \\
 \hline
 6 & & & & & & & \cellcolor{gray} & & & & \\
 \hline
 7 & & & & & & & \cellcolor{gray} & \cellcolor{gray} & \cellcolor{gray} & & & \\
 \hline
 8 & & & & & & & & \cellcolor{gray} & \cellcolor{gray} & \cellcolor{gray} & & \\
 \hline
 9 & & & & & & & & & & \cellcolor{gray} & \cellcolor{gray} & \cellcolor{gray} \\
 \hline
 10 & & & & & & & & & & & & \cellcolor{gray} \\
 \hline
\end{tabular}
\end{table}


\bibliography{bibliografia}
\bibliographystyle{abnt}

\chapter{Assinaturas}
\vspace{2cm}

\begin{center}
\rule{8cm}{.3mm}
\medskip

	Michael Alexandre Costa\\
	Proponente

\end{center}

\vspace{4cm}

\begin{center}
\rule{8cm}{.3mm}
\medskip

	André R. Du Bois\\
	Prof. Orientador

\end{center}
\end{document}
